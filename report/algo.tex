\section{Algorithmen} \label{sec:algo}
Dieser Abschnitt besch\"aftigt sich mit den entwickelten Algorithmen f\"ur das bilden von Clustern, die Ausfallsicherheit, und das Rotieren von Clusterheads.

\subsection{Clusterbildung}
Die Grundlage des Clusteringalgorithmus ist das Bilden von vollst\"andigen Graphen.
Die Knoten des Graphen sind die Sensoren, und die Kanten die W-Lan Verbindungen in Reichweite.
Dadurch reduziert sich die Clusterbildung auf das Cliquenproblem.

Cluster werden durch folgendes Protokol gebildet:
\begin{itemize}
\item Falls sich ein Sensor aktiviert, so sendet sie zuerst eine Nachricht die nach vorhandenen Clustern sucht. Findest der Sensor keine vorhandenen Cluster, so bildet er selber einen.
\item Alle schon vorhandenen Clusterheads antworten auf die Anfragen von neuen Motes. Diese Antwort enth\"allt einen Identifikator des Clusters und die Anzahl der Sensoren in dem Cluster.
\item Der neue Sensor speichert alle Antworten der vorhanden Cluster, ordnet sie nach Gr\"o\ss e und versucht der Reihe nach einem der Cluster beizutreten, beginnend mit dem Kleinsten.
\item Der erste Schritt zum Beitreten eines Clusters, das Senden eine Nachricht, auf die alle Mitglieder des Clusters mit ihrer Id antworten.
\item Nach dem Ablaufen eines Timeouts, sendet der neue Sensor die Ids aller empfangenen Sensoren an den Clusterhead. Dies stellt sicher, dass der neue Sensor alle schon vorhandenen Mitglieder erreichen kann.
\item Falls die Nachricht des neuen Sensors alle Ids des aktuellen Clusters entahlten, so sendet der Clusterhead dem neuen Sensor eine Nachricht mit der Best\"atigung, dass er neue Sensor dem Cluster beigetreten ist. Zus\"atzlich ordnet der Clusterhead dem neuen Sensor einen Slot zu. Dieser Slot wird n\"otig, falls der Clusterhead ausf\"allt.
\item Falls die Nachricht des neuen Sensors nicht alle Ids enthalten sollte, so sendet der Server eine Ablehnung und der Client versucht dem n\"achst gr\"o\ss eren Cluster beizutreten.
\end{itemize}

\begin{figure*}
\begin{tikzpicture}[->,>=stealth',shorten >=1pt,auto,node distance=4cm,
                    semithick]
  \tikzstyle{every state}=[circle,fill=black!25, minimum size=5em]

  \node[initial,state] (A)              {\shortstack{WhoIs\\There}};
  \node[state]         (B) [right of=A] {\shortstack{List\\Members}};
  \node[state]         (C) [right of=B] {\shortstack{Join\\Request}};
  \node[state]         (D) [right of=C] {\shortstack{Cluster\\Member}};
  \node[state]         (E) [below of=D] {\shortstack{Cluster\\Head}};

  \path (A) edge [bend right=10]  node {Timeout} (E)
            edge [bend left]  node {\shortstack{WhoIsThere/\\ReWhoIsThere}} (B)
        (B) edge [bend left]  node {\shortstack{ListMembers/\\ReListMembers}} (C)
        (C) edge [bend left]  node {\shortstack{JoinRequest/\\ReJoinRequest(true)}} (D)
            edge [bend left] node {\shortstack{JoinRequest/\\ReJoinRequest(false)}} (B)
        (D) edge [loop above] node {\shortstack{ListMembers/\\ReListMembers}} (D)
            edge [bend left=10]  node {--/Rotate} (E)
        (E) edge [bend left=10]  node {Rotate/--} (D)
            edge [bend left=25] node {--/HeadGone} (A);
\end{tikzpicture}
\label{fig:sm}
\caption{State Machine des Cluster Protokolls \textbf{TODO}}
\end{figure*}
\subsection{Ausfallsicherheit} \textbf{TODO}
Aufgrund der erw\"ahnten Annahmen (TCP/IP, keine Bewegung der Motes, etc) gibt es nur zwei Ausfallszenarien die betrachtet werden m\"ussen, der Ausfall einer Membermote,  sowie der Ausfall des Clusterheads.
Falls eine Membermote ausfallen sollte, so muss der zugeh\"orige Clusterhead dies merken und die Liste der zum Cluster zugeh\"origen Motes aktualisieren.
Da Motes die dem Cluster betreten wollen Kontakt zu allen Motes nachweisen m\"ussen, die der Cluster als zum Cluster zugeh\"orig gespeichert hat, w\"urde der unbemerkte Ausfall einer Mote dazu f\"uhren, dass keine neuen Motes beitreten k\"onnen.
In einem realen Szenario erkennt der Clusterhead das Ausfallen einer Membermote an dem Fehler von gesendeten Sensordaten. In der nie vorgestellten Simulation wird durch das L\"oschen direkt beim Server ein L\"oschevent.

Der Ausfall von einem Clusterhead wird durch das Fehlen von Best\"atigungsnachrichten erkannt.
Aufgrund unserer Annahme geschickt dies bei allen Membermotes gleichzeitig.
Sobald eine Membermote den Ausfall des Clusterheads wahrnimmt, so wird der Cluster als tot angenommen.
Die Mote wartet nun eine bestimmte Zeit $b$ und versucht dann einem neuen Cluster beizutreten.
$b$ berechnet sich durch die upper bound f\"ur das Beitreten einer Mote zu einem Cluster multipliziert mit dem ``slot'' welcher der Mote beim Beitreten des jetzt toten Clusters zugewiesen wurde.
Der slot ist ein streng monoton ansteigender pro Cluster eindeutiger Integer.
Somit wird sichergestellt, dass die Membermote alle nacheinander in der urspr\"unglichen Beitrittsreihenfolge versuchen einem neuen Cluster beizutreten.

\subsection{Rotation der Clusterheads} \textbf{TODO}
Die Rotation des Clusterhead ist der Prozess welcher von dem aktuellen Clusterhead ausgef\"uhrt wird, und einer Membermote die F\"uhrung des Clusters zuweist.
Dieser Rotation kann ausgel\"ost werden von z.B. einem timeout oder dem Batteriestand.
Beider Durchf\"uhrung der Rotation w\"ahlst der Clusterhead der neuen Cluster und sickt alle relevanten Daten an ihn.
Die gesendeten Daten sind:
slot\\
Membermotes\\
...\\
