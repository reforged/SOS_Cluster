\section{Funktionale Anforderungen} \label{sec:features}
\begin{enumerate}
\item \textbf{Implementierung des beschriebenen Protokoll}
  \begin{itemize}
  \item \emph{Beschreibung:}\\
    Das Protokoll bietet die Hauptfunktionalit\"at f\"ur das Anwendungszenario. Eine vollst\"andige und fehlerfreie Implementierung ist daher \emph{kritisch}.
  %\item \emph{Problembeschreibung:}\\

  \item \emph{Siehe:}\\
    Abschnitt \ref{sec:algo} f\"ur die Spezifikation des Protokolls.
  \item \emph{Abnahmekriterium:}\\
    Diese Anforderung gelte als erf\"ullt, wenn die Implementierung folgendes leistet: das Bilden von Verb\"unden, das Behandeln von Ausf\"allen von Sensorknoten und die Rotation von Clusterheads anhand des genannten Protokolls.
  \end{itemize}

\item \textbf{Kapselung der Komponenten}
  \begin{itemize}
  \item \emph{Beschreibung:}\\
  Einzelne Systemponenten sind auf realistische Weise zu kapseln.
  So dürfen Mote-Objekte untereinander nur durch Nachrichtenversenden kommunizieren, und keine Informationen über ihre Lage innerhalb  der Simulation erlangen können. 
  \item \emph{Problembeschreibung:}\\
Um eine m\"oglichst realistische Simulation zu sein, und um als echter Proof-Of-Concept zu gelten, m\"ussen die simulierten Komponenten voneinander entkoppelt sein, also \"uber Schnittstellen kommunizieren, welche auch in einem Anwendungszenarion bestehen w\"urden.
  \item \emph{Abnahmekriterium:}\\
    Zur Erf\"ullung dieser Anforderung sind n\"otig: Kapselung der Mote-Objekte untereinander, Kapselung der Mote-Objekte von der Simulationsumgebung und ein Mechanismus zum Nachrichtenversenden.
  \end{itemize}
\item \textbf{Graphische Oberfl\"ache}
  \begin{itemize}
  \item \emph{Beschreibung:}\\
    Die Implementierung hat eine graphische Benuzteroberfl\"ache zur Verf\"ugung zu stellen, welche die Simulation visualisiert. Die Bedienung der Oberfl\"ache hat ein sinnvolles Subset der Gesamtfunkionalit\"at zu implementieren.
  \item \emph{Problembeschreibung:}\\
  Zur Präsentation der Implementierung ist eine graphische Oberfläche unentbehrlich.
  \item \emph{Abnahmekriterium:}\\
    Die Implementierung muss eine graphische Benuzteroberfl\"ache haben, welche die simulierte Umgebung, samt Sensoreknoten darstellt. Gebildete Cluster m\"ussen farblich markiert sein. Zus\"atzlich soll man Sensorknoten hinzuf\"ugen und entfernen k\"onnen.
  \end{itemize}
\end{enumerate}
