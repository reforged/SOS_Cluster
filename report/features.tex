\section{Funktionale Anforderungen} \label{sec:func}
\begin{enumerate}
\item \textbf{Clusterbildung nach dem beschriebenen Protokoll}
  \begin{itemize}
  \item \emph{Beschreibung:}\\
    Die Implementation hat aus Mote-Objekten Cluster zu bilden. Die Bildung der Cluster hat nach dem in Abschnitt \ref{sec:algo} beschriebenen Algorithmus zu erfolgen.
  \item \emph{Problembeschreibung:}\\
    Prim\"ares Ziel des Projektes ist das Bilden von Clustern von Sensorknoten. Die Wichtigkeit dieses Punktes ist daher \emph{kritisch} und selbsterkl\"arend.
  \item \emph{Abnahmekriterium:}\\
    Diese Anforderung gelte als erf\"ullt, sobald die Mote-Objekte der Implementation bereitstellt, die das Protokoll zur Formung von Clustern vollst\"andig implementieren.
  \end{itemize}
\item \textbf{Rotation des Clusterheads nach dem beschriebenen Protokoll}
  \begin{itemize}
  \item \emph{Beschreibung:}\\
    Die durch Punkt 1 gebildeten Cluster haben einen Clusterhead. Die Implementation hat den Algorithmus zur Rotation der Clusterheads, welcher auch in Abschnitt \ref{sec:algo} beschrieben ist, zu implementieren.
  \item \emph{Problembeschreibung:}\\
    Eine der Priorit\"aten bei dem Design des Protokolles zum Bilden von Clustern war die Austauschbarkeit der Clusterheads. Durch die Reduktion auf vollst\"andige Graphen ist die einfach zu erreichen. Die Rotation ist daher ein guter Showcase f\"ur die Vorz\"uge des Protokolls zum Bilden von Clustern.
  \item \emph{Abnahmekriterium:}\\
    Diese Anforderung gelte als erf\"ullt, wenn die gebildeten Cluster den Clusterhead nach dem spezifizierten Protokoll implentiert.
  \end{itemize}
\item \textbf{Kapselung der Komponenten}
  \begin{itemize}
  \item \emph{Beschreibung:}\\
    Einzelne Systemkomponenten haben auf realistische Weise voneinander gekapselt sein.
    Dies bezieht sich insbesondere, aber nicht aussschlie\ss lich auf den Motes zur Verf\"ugung stehenden Informationen, wie Lage, Informationen \"uber andere Sensorknoten, sowie Zustellung der Kommunikationen unter dem Motes durch einen von dem Sensorknoten unabh\"anigen Mechanismus.
  \item \emph{Problembeschreibung:}\\
Um eine m\"oglichst realistische Simulation zu sein, und um als echter Proof-Of-Concept zu gelten, m\"ussen die simulierten Komponenten voneinander entkoppelt sein, also \"uber Schnittstellen kommunizieren, welche auch in einem Anwendungszenarion bestehen w\"urden.
  \item \emph{Abnahmekriterium:}\\
    Zur Erf\"ullung dieser Anforderung sind n\"otig: Kapselung der Mote-Objekte untereinander, Kapselung der Mote-Objekte von der Simulationsumgebung und ein Mechanismus zum Nachrichtenversenden.
  \end{itemize}
\item \textbf{Graphische Oberfl\"ache}
  \begin{itemize}
  \item \emph{Beschreibung:}\\
    Die Implementation hat eine graphische Benuzteroberfl\"ache zur Verf\"ugung zu stellen, welche die Simulation visualisiert. Die Bedienung der Oberfl\"ache hat ein sinnvolles Subset der Gesamtfunkionalit\"at zu implementieren.
  \item \emph{Problembeschreibung:}\\
    Um die Simulation zu visualisieren ist eine graphische Oberfl\"ache unentbehrlich. Sie dient zur demonstration des Proof-Of-Concepts. Das Hinzuf\"ugen und Entfernen von Motes erlaubt das Demonstrieren des vollen Umfanges der Funktionalit\"at.
  \item \emph{Abnahmekriterium:}\\
    Die Implementation muss eine graphische Benuzteroberfl\"ache haben, welche die simulierte Umgebung, samt Sensoreknoten darstellt. Gebildete Cluster m\"ussen farblich markiert sein. Zus\"atzlich soll man Sensorknoten hinzuf\"ugen und entfernen k\"onnen.
  \end{itemize}
\item \textbf{Ausfallsicherheit?}
\end{enumerate}
