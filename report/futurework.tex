\section{Future Work} \label{sec:futwork}
Die zur Verf\"ugung gestellte Implementierung des beschriebenen Protokolls erf\"ullt die funktionalen Anforderungen im vollen Ma\ss e.
Um aber den Einsatz des Protokolles realistischer zu machen, ist noch Arbeit an den Annahmen aus Abschnitt \ref{sec:setting} n\"otig.
So w\"are s m\"oglich das Protokoll um die M\"oglichkeit zu erweitern mehreren Sensorknoten zu erlauben gleichzeitig einem Cluster beizutreten.
Dies w\"are zum Beispiel realisierbar, in dem der Clusterhead bei einem schon laufenden join-Vorgang blockiert, und der Sensorknoten wartet und danach erneut versucht dem Cluster beizutreten.

Zus\"atzlich m\"usste das Protokoll auf in der Realität genutze drahtlose Kommunikationprotokolle, die unter anderem kein Multicast unterst\"utzen, angepasst werden.

\section{Fazit} \label{sec:conclusion}
Dieser Bericht stellt ein Protokoll vor, welcher das Bilden von  Sensorknoten-Clustern erlaubt. Es unterst\"utzt neben dem Bilden von Clustern auch den Ausfall von Knoten, sowie die Rotation des Clusterheads.
Bereitgestellt wird zus\"atzlich eine Implementation des Protokolls in Form einer Simulation. Umfang der Simulation ist neben einer graphischen Darstellung auch ein gekapseltes Simulationsumfelt, und die Möglichkeit Motes hinzuzufügen, oder zu entfernen.
