\section{Implementierung} \label{sec:impl}

Um die dargelegten Algorithmen zu testen und zu präsentieren, haben wir
eine Simulation entwickelt. Im folgenden wird zunächst die gewählte
Entwicklungsplattform vorgestellt und begründet, anschließend werden die
Implementierung des Äthers, der Sensorknoten und die Spezifika der
Simulations-Umgebung dargestellt.

\subsection{JavaScript}

Die Simulation ist in JavaScript implementiert. Als
Entwicklungsplattform haben wir Webbrowser mit aktueller
JavaScript-Engine (z.B. Googles V8 und Mozillas JägerMonkey) und
Unterstützung für 2D-Grafikdarstellung mithilfe des canvas-Tags gewählt.

Mit JavaScript als dynamisch typisierter, prototypenbasierter
Skriptsprache lassen sich ohne großen Deklarations-Overhead schnell
sichtbare Ergebnisse produzieren. Mit dem neuen HTML5 Canvas Element
steht eine einfach zu nutzende Zeichenfläche zur Verfügung und für alle
gängigem Betriebssysteme gibt es Browser, welche die Simulation auführen
können.

Obwohl JavaScript eine klassenlose Sprache ist, werde ich ähnliche
Termini wie in anderen objektorientierten Sprachen (also auch den
Terminus ``Klasse'') verwenden.

\subsection{Sensorknoten}

Die simulierten Sensorknoten sind Instanzen der Klasse ``Mote'', deren
Implementierung sich über die Dateien mote.js, mote\_member.js und
mote\_head.js erstreckt.

Das in Abschnitt \ref{sec:algo} dargestellte Protokoll ist vollständig, aber
nicht als Finite-Sate-Machine, in der Klasse Mote implememtiert.

Wird ein neuer, simulierter Sensorknote von der Klasse Mote instanziiert,
so initialisiert der Konstruktur einige interne Variablen und registriert
den neuen Sensorknoten beim Äther; anschließend kann dieser gestartet
(``eingeschaltet'') werden.

\subsection{Äther}

Die simulierten Sensorknoten (``Motes'') kommunizieren untereinander nur über
Nachrichten. Den (virtuellen) Versand der Nachrichten ist implementiert im
Modul \footnote{\url{http://yuiblog.com/blog/2007/06/12/module-pattern/}}
``MoteList'' in der Datei motelist.js.

Ähnlich dem Observer-Entwurfsmuster stellt MoteList eine Methode
``register'' bereit, die neuer Sensorknoten aufnimmt. Sie wird vom
Mote-Konstruktur aufgerufen. Zudem gibt es eine Methode ``send'', die
den Versandt von Nachrichten anstößt.

Beim Versandt wird zunächst die Position des Absenders abgerufen,
anschließend wird die Distanz zu allen anderen registrierten
Sensorknoten berechnet. Ist die Distanz nicht größer als ein
festgelegter Senderadius, wird die Nachricht zugestellt indem sie der
``onRecv'' Methode des jeweiligen Empfänger-Knotens übergeben wird. Der
Absender erhält seine eigenen Nachrichten nicht.

\subsection{Nachrichten}

Bei den übergebene Nachrichten handelt es sich um JavaScript-Objekte.
Sie sind nicht Instanzen einer bestimmten Klasse sondern nur des
Basis-Typs Object, somit folgen sie auch keinem strikten Format.

Alle Nachrichten haben jedoch per Konvention eine Eigenschaft ``type'',
die Aufschluss über die Intention der Nachricht gibt (z.B. WHOISTHERE,
JOINREQ, ROTATE) und somit auch die Kodierung der übrigen Eigenschaften
impliziert.

Über eine Netzwerk könnten die Nachrichten JSON-kodiert verschickt
werden, in der Simulation verzichten wir jedoch zugunsten der
Performance auf das Marshalling und Unmarshalling beim Absender
respektive Empfänger.

\subsection{Zeit}

Um die Geschwindigkeit der Simulation regulieren zu können sind alle
relevanten Zeitfenster (also insbesondere Timeouts) an die globale
Variable timeScale gekoppelt.

\subsection{Graphische Oberfläche}

Die graphische Oberfläche beschränkt sich im wesentlichen auf eine
``Karte'' in der alle Motes, möglichen Verbindungen, sowie Cluster
verzeichnet sind.

Die Zeichenfunktion ist dabei im Modul MoteList implementiert, da nur
darin Zugriff auf die Positionen der Sensorknoten möglich ist.

Die Sensorknoten sind als kleine Quadrate auf der Karte dargestellt,
oberhalb eines Cluserheads wird jeweils die ID des Clusters angezeigt.
Zwischen Sensorknoten die nicht mehr als einen Senderadius voneinander
entfernt sind, ist eine blasse Linie eingezeichnet. Zwischen zwei Knoten
die zu einem Cluster gehören, wird diese Linie farblich hervorgehoben.

Weitere Sensorknoten werden hinzugefügt, wenn man auf eine freie Fläche
der Karte klickt. Klickt man einen vorhandenen Knoten an, wird dieser
entfernt.

Die Rotation der Cluster-Heads wird durch Drücken der Taste 'r'
ausgelöst. Die Taste 'a' schaltet die automatische Rotation der
Cluster-Heads in einem fixen Intervall ein oder aus.

\section{Analyse} \label{sec:ana}

Im folgenden wollen wir überprüfen, ob die Implementierung die in
Abschnitt III formulierten Anforderungen erfüllt und die Performance der
Simulation untersuchen.

\subsection{Implementierung des Protokolls}

Die Bildung von Clustern ist in den simulierte Sensorknoten gemäß der
Spezifikation in Abschnitt \ref{sec:algo} implementiert. In der Simulation können
Sensorknoten sowohl automatisch, als auch manuell platziert werden und
die Knoten schließen sich dann zuverlässig einem Cluster an oder eröffnen
einen neuen.

Entsprechend der Spezifikationen erkennen Sensorknoten das Verschwinden
anderer Knoten nicht selber, reagieren aber wie im Protokoll festgelegt,
sobald sie über den Ausfall informiert werden.

Die Rotation des Clusterheads erfolgt bei Clustern mit mehreren Knoten
ebenfalls nach Protokoll, jedoch wird (entsprechend der Spezifikation)
nur nach externem Impuls rotiert und die Funktion des Clusterheads
übernimmt ein zufällig ausgewählter, anderer Sensorknoten.

\subsection{Kapselung der Komponenten}

Eine grundlegende Kapselung der simulierten Sensorknoten untereinander
besteht bereits darin, dass die Knoten jeweils als Instanzen der Klasse
Mote realisiert sind.

Referenzen auf die Sensorknoten (also die Instanzen der Klasse Mote)
werden nur innerhalb des MoteList Modules gespeichert. Die Liste der
Sensorknoten ist im Modul-Entwurfsmuster der MoteList durch den
Variablen-Skopus geschützt und nur Funktionen dieses Moduls können
darauf zugreifen. Der direkte Zugriff zwischen Sensorknoten ist somit
per Design ausgeschlossen.

Zusammen mit der Liste der Sensorknoten werden auch die Positionen der
Knoten im Modul MoteList geschützt gespeichert. Damit sind alle
relevanten Informationen der Simulationsumgebung vor den Sensorknoten
abgekapselt.

Die Kommunikation der Sensorknoten untereinander läuft somit
ausschließlich über den Nachrichtenversand, der im Äther (Modul
MoteList) implementiert ist. Nachrichten werden unter Berücksichtigung
eines Senderadius nur an ein Subset aller Sensorknoten übermittelt.

\subsection{Graphische Oberfläche}

Wie in den funktionalen Anforderungen gefordert, wird in der
Kartenansicht die simulierte Umgebung mit den darin enthaltenen
Sensorknoten dargestellt. Cluster werden durch farbige Linien zwischen
allen Knoten des Clusters als zusammenhängende Einheit erkennbar
gemacht.

Sensorknoten können hinzugefügt und entfernt werden.

\subsection{Performance}

Eine gute Performance war nicht Ziel dieser Simulation und ist im
Ergebnis auch nur für Demonstrationszwecke ausreichend.

Probleme treten auf, wenn die Nachrichten-Verteilung nicht schnell genug
erfolgen kann und Timeouts der Sensorknoten ablaufen, bevor die
Nachrichten zugestellt wurden. Das führt meistens dazu, dass
Sensorknoten nicht von vorhandenen Clustern erfahren oder nicht
rechtzeitig alle Mitglieder eines Clusters auflisten können, somit der
Beitritt zu vorhanden Clustern fehlschlägt und unnötig neue Cluster
gegründet werden.

Wieviele Sensorknoten ohne verspätete Zustellung von Nachtichten
bewältigt werden können, hängt stark von der gewählten
Simulationsgeschwindigkeit, sowie der der verwendeten JavaScript-Engine
und natürlich der Rechenleistung eines CPU-Kerns ab. Ebenso beeinflusst
die Verteilung der Sensorknoten und der Senderadius die Performance:
Liegen die Sensorknoten stark geballt oder ist der Senderadius groß
gewählt, werden alle Nachrichten von vielen Knoten empfangen und das
System kommt schneller an seine Grenzen, als bei weit verteilten
Sensorknoten und kleinen Senderadien.

Bei den Standardeinstellungen (Karte mit 780x1248 Positionen,
Senderadius von 15\% der Kartenhöhe, 150 Sensorknoten) werden alleine
800 Nachrichten verschickt und rund 3000 bis 3300 Nachrichten empfangen.

Platziert man dagegen nur 30 Knoten so dicht beieinander, dass sie ein
einzelnes Cluster bilden, werden dabei bereits 551 Nachrichten
verschickt und 10.294 Nachrichten empfangen. Zwingt man dieses Cluster
zur Neubildung, indem man den Clusterhead löscht, entstehen dabei
innerhalb von 5 Sekunden nochmal 519 Nachrichten die 14.533 mal
empfangen werden.

Die obenstehenden Versuche wurden auf einem Core2Duo Mobil-Prozessor mit
1,87GHz Kerngeschwindigkeit in Googles Browser Chrome mit der
V8-JavaScript-Engine durchgeführt. Während der Clusterneubildung wurde
ein Prozessorkern für 5 Sekunden voll ausgelastet.

Aufgrund der vielen Faktoren die die Performance beeinflussen soll hier
auf eine weitere Analyse der Performance auf verschiedenen Systemem
verzichtet werden.

Mit den Standardeinstellungen und in einem aktuellen Browser sollte es
problemlos möglich sein, die Simulation zu testen.
